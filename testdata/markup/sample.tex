\documentclass{article}
\usepackage{amsmath}
\usepackage{amssymb}
\usepackage{graphicx}
\usepackage{hyperref}
\usepackage{listings}

\title{Sample LaTeX Document}
\author{Test Author}
\date{\today}

\begin{document}

\maketitle

\begin{abstract}
This is a sample LaTeX document for testing the LaTeX chunker.
It demonstrates various sectioning commands and mathematical typesetting.
\end{abstract}

\tableofcontents

\section{Introduction}

LaTeX is a typesetting system widely used for scientific and technical documents.
This document demonstrates various LaTeX features.

\subsection{History}

LaTeX was created by Leslie Lamport in 1984 as a set of macros for TeX.
It has become the standard for academic publishing in many fields.

\subsection{Key Features}

LaTeX provides many powerful features:

\begin{itemize}
    \item Professional typography
    \item Mathematical equation support
    \item Automated numbering and cross-references
    \item Bibliography management
\end{itemize}

\section{Mathematics}

LaTeX excels at mathematical typesetting.

\subsection{Inline Math}

Inline equations like $E = mc^2$ are easy to write.
The quadratic formula is $x = \frac{-b \pm \sqrt{b^2 - 4ac}}{2a}$.

\subsection{Display Math}

For important equations, use display mode:

\begin{equation}
    \int_{-\infty}^{\infty} e^{-x^2} dx = \sqrt{\pi}
\end{equation}

The Euler identity is particularly beautiful:

\begin{equation}
    e^{i\pi} + 1 = 0
\end{equation}

\subsection{Multi-line Equations}

The align environment handles multi-line equations:

\begin{align}
    (x + y)^2 &= x^2 + 2xy + y^2 \\
    (x - y)^2 &= x^2 - 2xy + y^2 \\
    x^2 - y^2 &= (x+y)(x-y)
\end{align}

\subsection{Matrices}

Matrices are straightforward:

\begin{equation}
    \begin{pmatrix}
        a & b \\
        c & d
    \end{pmatrix}
    \begin{pmatrix}
        x \\
        y
    \end{pmatrix}
    =
    \begin{pmatrix}
        ax + by \\
        cx + dy
    \end{pmatrix}
\end{equation}

\section{Code Listings}

LaTeX can display code with the listings package:

\begin{lstlisting}[language=Python, caption=A Python Example]
def fibonacci(n):
    """Generate Fibonacci sequence up to n."""
    a, b = 0, 1
    result = []
    while a < n:
        result.append(a)
        a, b = b, a + b
    return result

print(fibonacci(100))
\end{lstlisting}

\subsection{Inline Code}

Use \texttt{verbatim} for inline code: \texttt{print("Hello, World!")}.

\section{Tables and Figures}

\subsection{Tables}

Here is a sample table:

\begin{table}[h]
\centering
\begin{tabular}{|l|c|r|}
    \hline
    Left & Center & Right \\
    \hline
    A1 & B1 & C1 \\
    A2 & B2 & C2 \\
    A3 & B3 & C3 \\
    \hline
\end{tabular}
\caption{A sample table}
\label{tab:sample}
\end{table}

\subsection{Figures}

Figures would be included as follows:

\begin{figure}[h]
\centering
% \includegraphics[width=0.5\textwidth]{sample-image}
\caption{A sample figure placeholder}
\label{fig:sample}
\end{figure}

\section{Theorems and Proofs}

\subsection{Theorem Environments}

\newtheorem{theorem}{Theorem}
\newtheorem{lemma}{Lemma}
\newtheorem{definition}{Definition}

\begin{definition}
A prime number is a natural number greater than 1 that has no positive divisors other than 1 and itself.
\end{definition}

\begin{theorem}[Euclid]
There are infinitely many prime numbers.
\end{theorem}

\begin{proof}
Assume there are finitely many primes $p_1, p_2, \ldots, p_n$.
Consider $N = p_1 p_2 \cdots p_n + 1$.
This number is not divisible by any $p_i$, so it must have a prime factor not in our list.
This contradicts our assumption.
\end{proof}

\section{Cross-References}

As shown in Table~\ref{tab:sample} and Figure~\ref{fig:sample}, LaTeX handles cross-references automatically.

\subsection{Citations}

Citations use BibTeX: According to~\cite{lamport1994latex}, LaTeX is powerful.

\section{Advanced Topics}

\subsection{Custom Commands}

Define custom commands for convenience:

\newcommand{\R}{\mathbb{R}}
\newcommand{\N}{\mathbb{N}}

The real numbers $\R$ and natural numbers $\N$ are fundamental sets.

\subsection{Environments}

Custom environments provide structure:

\newenvironment{example}
{\begin{quote}\textbf{Example:}}
{\end{quote}}

\begin{example}
This is an example environment.
\end{example}

\subsubsection{Nested Sections}

LaTeX supports deep nesting of sections.

\paragraph{Paragraphs}

Paragraphs are even more specific than subsubsections.

\subparagraph{Subparagraphs}

Subparagraphs are the deepest standard sectioning level.

\section{Conclusion}

This document demonstrates various LaTeX features for testing purposes.
The chunker should properly identify section boundaries and preserve equations.

\begin{thebibliography}{9}
\bibitem{lamport1994latex}
    Leslie Lamport,
    \textit{LaTeX: A Document Preparation System},
    Addison-Wesley, 1994.
\end{thebibliography}

\end{document}
